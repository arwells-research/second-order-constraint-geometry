\section{Empirical Case Studies Relevant to $\Sigma_2$}
\label{sec:empirical-case-studies-sigma2}

This section surveys representative experimental classes in which observable
behavior is difficult to reconcile with state-sufficient descriptions, but arises
naturally once admissibility is treated as a property of histories rather than
instantaneous states. The goal is not to reinterpret quantum mechanics, but to
identify operational regimes in which state-based sufficiency fails.

Each case considered here is empirically robust, widely accepted within standard
quantum mechanics, and diagnostically useful because it isolates ordering or
trajectory dependence without introducing new dynamics or speculative mechanisms.

\subsection{Delayed-choice and retroactive conditioning experiments}

Delayed-choice experiments probe whether present measurement choices constrain
which histories are admissible, even when intermediate system descriptions are
identical across experimental branches. At the level of reduced state, the system
admits the same instantaneous description at the conditioning boundary, yet
subsequent choices alter the set of admissible continuations. Observable statistics
then diverge despite the absence of any state-local distinction at earlier times.

From a $\Sigma_2$ perspective, the diagnostic content of these experiments lies in
this mismatch between state equivalence and admissibility equivalence. The behavior
violates strict state sufficiency while remaining compatible with globally
constrained admissible histories, without invoking retrocausation or modified
dynamics.

\subsection{Postselection-induced statistical structure}

Postselected ensembles are common in quantum optics and weak measurement contexts.
Operationally, postselection generates ensembles whose statistics cannot be
reproduced by any convex mixture of forward-prepared states sharing the same
instantaneous description. This remains true even when disturbance and selection
bias are carefully controlled.

Within $\Sigma_2$, postselection is understood not as filtering states, but as
restricting admissible histories. The resulting statistical structure reflects
constraints on trajectory continuation rather than properties encoded in the
instantaneous system state. The failure of state sufficiency in this setting is
therefore operational and structural, not interpretive.

\subsection{Decoherence with history-dependent envelopes}

Standard decoherence theory assumes that reduced dynamics depend only on
instantaneous system--environment couplings. Empirically, however, decoherence
envelopes often depend on features of prior interaction history, including the
ordering and duration of earlier coupling regimes, even when reduced states and
instantaneous Hamiltonians are matched.

Such effects are commonly described as non-Markovian, presupposing an underlying
state description augmented by memory variables. $\Sigma_2$ offers a different
classification: admissible system--environment trajectories are globally
constrained, and this constraint persists even when no localized memory register
can be identified. The observed dependence therefore reflects a limitation of
state sufficiency rather than a failure of dynamical modeling.

\subsection{Interference recovery and which-path erasure}

Quantum eraser experiments demonstrate that interference can be recovered after
which-path information appears to have been irreversibly dispersed. In these
experiments, reduced system states can be operationally identical in cases with and
without recovered interference, while observable outcomes nevertheless differ.

From a second-order perspective, this distinction arises from admissible
correlation histories rather than from any restoration of instantaneous state
information. Interference recovery depends on trajectory-consistent erasure
conditions, not on state reconstruction. This directly exhibits failure of state
sufficiency without invoking hidden variables or retrocausal influence.

\subsection{Bell-type experiments and ordering dependence}

Bell inequality violations are often discussed in terms of nonlocal or contextual
state properties. However, Bell-type experiments also exhibit sensitivity to
measurement ordering, detector timing, and postselection criteria. These
sensitivities persist even when reduced states are operationally matched and cannot
be eliminated by state augmentation without introducing additional asymmetry.

$\Sigma_2$ reframes this behavior as arising from constraints on joint admissible
histories across spacelike-separated regions, rather than from violations of local
state realism. Importantly, this classification predicts no deviation from quantum
correlations. It predicts instead that attempts to restore state sufficiency via
hidden variables, retrocausal states, or enlarged ontologies encounter structural
obstructions tied to history-level admissibility.

\subsection{Macroscopic order-sensitive protocols}

Order dependence without state dependence is not confined to microscopic systems.
In controlled macroscopic protocols, identical initial and final states, identical
energy budgets, and identical instantaneous observables can nevertheless yield
distinct outcomes depending solely on the ordering of operations.

Such phenomena are often described as path dependence or hysteresis. Within
$\Sigma_2$, they are understood as macroscopic instances of the same structural
feature observed in microscopic settings: admissibility depends on trajectory even
when state-level descriptions agree. The distinction is therefore one of
representational regime, not of scale.

Across these diverse cases, a common diagnostic pattern emerges. Instantaneous
state descriptions fail to encode all operationally relevant constraints, while
history-level admissibility remains decisive. $\Sigma_2$ provides a unified
geometric language for this failure without introducing new dynamics or modifying
empirical predictions.