\section{Toy Examples of Second-Order Constraint Geometry}
\label{sec:toy-examples-sigma2}

This section presents minimal, abstract examples illustrating diagnostic features
associated with second-order constraint geometry $\Sigma_2$, without reference to
any specific physical system. The purpose is not to establish irreducible
second-order structure, nor to claim the impossibility of faithful state
representation in general, but to make explicit how history-dependent
admissibility can arise and how it manifests under state-based reduction.

These examples are deliberately simplified. They function pedagogically, not as
physical counterexamples, and are intended to illustrate patterns that motivate
the diagnostic criteria developed in Sec.~\ref{sec:definition-sigma2}. Whether
any particular toy constraint requires a second-order description in the sense
of Definition~\ref{def:faithful-state-representation} is not asserted here.

\subsection{Order-sensitive admissibility}

Consider a system whose admissible histories consist of sequences of operations
drawn from the set $\{A,B\}$, subject to the rule that operation $B$ is admissible
only if it has been preceded by $A$.

This constraint does not restrict the instantaneous availability of $A$ or $B$,
nor does it impose any condition on a local state variable. Admissibility depends
solely on ordering within a history. The sequence $(A,B)$ is admissible, while
$(B,A)$ is not, despite both containing the same operations.

Any reduced description that records only which operations occur, but not their
order, fails to distinguish these histories. Admissibility is therefore not
state-local, but relational across history segments. This illustrates a basic
order-sensitive admissibility pattern characteristic of $\Sigma_2$ diagnostics.

\subsection{Degeneracy under state-only reduction}

Let $\mathcal{H}$ denote the space of all finite sequences over $\{A,B\}$, and let
$\Pi$ map each history to the count vector $(n_A,n_B)$.

Under this reduction, the admissible history $(A,B)$ and the inadmissible history
$(B,A)$ are operationally indistinguishable. The reduction collapses distinct
admissibility classes into a single state description.

This degeneracy reflects loss of ordering information under state-only reduction.
It does not, by itself, establish irreducible state insufficiency, but illustrates
how admissibility distinctions can be erased when history-level relations are
discarded.

\begin{remark}
Augmenting the state space with a variable indicating whether $A$ has occurred
restores admissibility discrimination. This shows that state augmentation is
possible in this toy system. However, the augmentation functions by explicitly
encoding ordering information, thereby relocating history dependence into the
state description rather than eliminating it.
\end{remark}

\subsection{Boundary-conditioned admissibility}

Consider a variant in which admissibility is defined relative to a reference
boundary: a history is admissible if the number of $A$ operations preceding the
boundary equals the number of $B$ operations following it.

This constraint depends jointly on pre- and post-boundary segments of a history.
It cannot be evaluated at any single point along the sequence without reference
to the boundary location. Any operational reduction that lacks access to this
boundary information must either fix a boundary by convention, discard
admissibility distinctions, or introduce an explicit asymmetry between segments.

This example illustrates how boundary-conditioned admissibility can generate
diagnostic failures under reduction, without asserting that faithful state
representation is impossible in all realizations.

\subsection{Order-sensitive transformation composition}

Define two transformations on histories: $T_1$, which appends operation $A$, and
$T_2$, which appends operation $B$.

Under the constraint that $B$ requires a prior $A$, the composition
$T_2 \circ T_1$ maps admissible histories to admissible histories, while
$T_1 \circ T_2$ does not. Both compositions are well-defined as sequences, but
only one preserves admissibility.

This non-commutativity arises from admissibility structure rather than from
algebraic properties of the transformations themselves. It serves as a simple
illustration of order sensitivity, not as a proof of irreducible second-order
structure.

\subsection{Limits of symmetric state augmentation}

One may attempt to restore state sufficiency by enlarging the state space to store
information about prior operations. In the present toy examples, such
augmentation is straightforward and effective.

The diagnostic point is not that augmentation fails, but that it succeeds by
encoding history-dependent relations explicitly. Whether such encoding preserves
faithfulness in the sense of Definition~\ref{def:faithful-state-representation}
depends on whether it respects the symmetry structure of the admissible-history
space in question.

\begin{remark}
These toy examples do not establish the impossibility of faithful state
representation. They illustrate why faithfulness must be assessed rather than
assumed when admissibility depends on ordering, boundary conditions, or relational
history structure.
\end{remark}

Taken together, these examples demonstrate how history-dependent admissibility
can be obscured by state-only descriptions and how such obscuration produces
diagnostic signatures associated with $\Sigma_2$. Whether these signatures
indicate genuine second-order constraint geometry in a physical system is an
empirical question, addressed in subsequent sections.