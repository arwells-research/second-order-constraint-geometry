\section{Implications for Foundations and No-Go Theorems}
\label{sec:implications-no-go}

Modern foundations of physics are marked by a proliferation of no-go theorems:
results that prohibit entire classes of explanations rather than supplying
constructive alternatives. Bell nonlocality, Kochen--Specker contextuality,
no-cloning, no-broadcasting, and related results are typically treated as
independent constraints arising from distinct physical principles.

This section argues that many such results share a common structural source. They
arise when admissibility becomes history-dependent while the theory continues to
be represented in a state-sufficient language. The present framework does not
replace existing no-go theorems, but clarifies why they recur and why attempts to
evade them by state-based augmentation repeatedly fail.

Most foundational no-go results rely implicitly on the assumption that all
physically relevant constraints are encoded in an instantaneous state. This
assumption is rarely stated explicitly, but it underwrites demands such as
context-independent value assignment, factorization of joint probabilities,
local determination of outcomes, and the independence of future behavior from
preparation history once the state is fixed. Second-order constraint geometry
identifies this assumption as state sufficiency.

From this perspective, Bell’s theorem does not merely exclude local hidden
variables. It reveals that correlations observed in entangled systems cannot be
generated by descriptions that treat admissibility as state-local. The violation
of Bell inequalities indicates that admissible joint histories cannot be
decomposed into independent state assignments, even when those states are
operationally well defined. The failure lies not in locality alone, but in the
assumption that joint outcomes can be determined from local states without
reference to shared history.

A similar structural diagnosis applies to contextuality. The Kochen--Specker
theorem shows that context-independent value assignment is impossible. Within the
present framework, this follows directly from the fact that admissibility depends
on the ordering and compatibility of measurements. Different measurement contexts
impose different constraints on admissible histories, even when they act on the
same underlying system. Once admissibility is trajectory-dependent, contextuality
is not anomalous; it is unavoidable.

The same logic extends to no-cloning and no-broadcasting results. Cloning an
arbitrary quantum state would require that a single admissible history branch
into multiple future histories that preserve identical phase relations under all
possible subsequent operations. Such duplication is prohibited because
admissibility is not closed under unrestricted branching. From the standpoint of
$\Sigma_2$, no-cloning reflects a constraint on admissible trajectory structure,
not merely an algebraic feature of Hilbert space.

The recurrence of impossibility results across disparate formalisms and
interpretive programs is therefore not accidental. It signals a persistent
mismatch between representational assumptions and physical structure. As long as
theories are forced to encode history-dependent admissibility using
instantaneous state descriptions, contradictions and no-go results will continue
to emerge. These theorems are not failures of particular models, but diagnostics
of a deeper structural limitation.

This diagnosis parallels the projection-induced arrow result developed in the
companion paper. There, arrow-like directedness arises when state sufficiency
fails under conditioning, even though the underlying dynamics are symmetric.
Here, foundational no-go theorems arise when state sufficiency fails under
contextual and historical constraints. Both results instantiate the same
principle: state-based representations reach their limits whenever admissibility
depends on history rather than on instantaneous configuration.

The framework advanced here does not assert that quantum mechanics is
inconsistent, defective, or empirically incomplete. Quantum theory remains
extraordinarily successful as an operational and predictive framework. The claim
is structural rather than critical. Quantum mechanics encodes second-order
constraints using first-order representational tools, and no-go theorems mark the
boundaries of that encoding.

By locating these results within a unified constraint geometry, second-order
constraint theory clarifies why adding hidden variables, modifying dynamics, or
altering ontologies so often relocates foundational problems rather than resolves
them. The payoff is not a new ontology, but a clearer map of which explanatory
moves are structurally possible once admissibility is recognized as a property of
histories rather than states.