\subsection{Falsification of the $\Sigma_2$ prediction}

The $\Sigma_2$ hypothesis makes a single falsifiable \emph{structural} claim:
within a fixed operational frame and under a boundary-symmetric reduction,
equality of reduced state at a conditioning boundary does not suffice to fix all
future reduced diagnostics when admissible history differs.

This claim is falsified if the following empirical condition holds.

\begin{quote}
\emph{For all admissible history pairs $H_A,H_B$ satisfying
$\rho_B^{(A)} = \rho_B^{(B)}$ under a boundary-symmetric operational reduction,
and for all subsequent evolution blocks $C$, the resulting reduced diagnostics are
statistically indistinguishable:
\[
D_A(\tau) = D_B(\tau)
\quad \text{for all probe parameters } \tau .
\]}
\end{quote}

Demonstrating this invariance across a sufficiently rich class of admissible
history constructions, conditioning procedures, and probe protocols would
establish that instantaneous reduced state suffices to characterize all
operationally accessible future behavior within the tested regime. In such a case,
no second-order constraint structure is required.

Conversely, the reproducible observation of history-dependent divergence
$D_A(\tau) \neq D_B(\tau)$ under verified boundary-state equality constitutes
direct evidence against state sufficiency. Under identical admissible dynamics and
post-boundary evolution, such divergence cannot be attributed to state-local
descriptions and therefore indicates irreducible trajectory dependence.

To distinguish genuine $\Sigma_2$ effects from finite-precision artifacts, the
protocol requires quantitative separation between diagnostic divergence and
boundary-state mismatch. Let $\varepsilon$ denote the verified upper bound on
state mismatch at the boundary (as determined by tomography or calibration), and
let $\Delta D(\tau) := |D_A(\tau) - D_B(\tau)|$. A $\Sigma_2$ signature is indicated
when
\[
\Delta D(\tau) \gg \mathcal{O}(\varepsilon)
\]
and persists under systematic refinement of boundary-state verification.

Operationally, this can be established by varying $\varepsilon$ through improved
control or post-selection while holding the admissible histories fixed. If
$\Delta D(\tau)$ extrapolates to a nonzero value as $\varepsilon \to$ its
experimental limit, the divergence cannot be attributed to incomplete state
matching.

The force of the test lies in its minimality. No modification of dynamics is
invoked, no ontological commitment to hidden variables is required, and no
environmental bookkeeping is assumed at the boundary. Any violation of the null
condition reflects a structural limitation of state-based descriptions within the
specified operational frame.