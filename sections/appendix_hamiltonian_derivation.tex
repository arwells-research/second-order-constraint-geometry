\section{Derivation Details for the Concrete Hamiltonian Example}
\label{app:hamiltonian-derivation}

This appendix provides the explicit density-matrix and partial-trace derivations
supporting Sec.~\ref{subsec:concrete-hamiltonian-example}.
We show (i) equality of reduced boundary state under two admissible histories
within the reduced operational frame, and (ii) divergence of a reduced diagnostic
under identical post-boundary evolution.

\subsection{Setup and notation}

Let $S$ and $E$ be qubits with computational basis $\{\ket{0},\ket{1}\}$.
Define $\ket{+} := (\ket{0}+\ket{1})/\sqrt{2}$.
Let
\begin{equation}
H_{SE}=\hbar g\,\sigma_z^{(S)}\otimes \sigma_z^{(E)}, \qquad
U(t)=e^{-\frac{i}{\hbar}H_{SE}t}=e^{-i\phi\,\sigma_z^{(S)}\otimes\sigma_z^{(E)}},
\qquad \phi:=gt.
\end{equation}
On the joint computational basis $\ket{ij}\equiv \ket{i}_S\ket{j}_E$,
$\sigma_z^{(S)}\otimes\sigma_z^{(E)}$ has eigenvalue $+1$ on $\ket{00},\ket{11}$ and
$-1$ on $\ket{01},\ket{10}$, hence
\begin{equation}
U(\phi)\ket{00}=e^{-i\phi}\ket{00},\;
U(\phi)\ket{01}=e^{+i\phi}\ket{01},\;
U(\phi)\ket{10}=e^{+i\phi}\ket{10},\;
U(\phi)\ket{11}=e^{-i\phi}\ket{11}.
\label{eq:Uphi_action}
\end{equation}

We take the initial product state
\begin{equation}
\ket{\psi_0}=\ket{+}_S\otimes\ket{+}_E
=\tfrac{1}{2}\big(\ket{00}+\ket{01}+\ket{10}+\ket{11}\big).
\label{eq:psi0}
\end{equation}

Write the first interaction duration as $t_\theta$ with $\theta:=gt_\theta$,
and the post-boundary block duration as $t_\phi$ with $\phi:=gt_\phi$.

\subsection{Boundary states for the two histories}

\paragraph{History $H_A$.}
After the first interaction segment,
\begin{align}
\ket{\psi_A(B)} &= U(\theta)\ket{\psi_0}
\nonumber\\
&=\tfrac{1}{2}\Big(e^{-i\theta}\ket{00}+e^{+i\theta}\ket{01}+e^{+i\theta}\ket{10}+e^{-i\theta}\ket{11}\Big).
\label{eq:psiA_B}
\end{align}

\paragraph{History $H_B$.}
Apply $(\mathbb{I}_S\otimes\sigma_x^{(E)})$ after the first interaction segment:
since $\sigma_x\ket{0}=\ket{1}$ and $\sigma_x\ket{1}=\ket{0}$,
\begin{align}
\ket{\psi_B(B)} &=
(\mathbb{I}_S\otimes\sigma_x^{(E)})\,\ket{\psi_A(B)}
\nonumber\\
&=\tfrac{1}{2}\Big(e^{-i\theta}\ket{01}+e^{+i\theta}\ket{00}+e^{+i\theta}\ket{11}+e^{-i\theta}\ket{10}\Big)
\nonumber\\
&=\tfrac{1}{2}\Big(e^{+i\theta}\ket{00}+e^{-i\theta}\ket{01}+e^{-i\theta}\ket{10}+e^{+i\theta}\ket{11}\Big).
\label{eq:psiB_B}
\end{align}

\subsection{Reduced boundary density matrix $\rho_S(B)$}

Define $\rho_{SE}^{(A)}(B)=\ket{\psi_A(B)}\!\bra{\psi_A(B)}$ and
$\rho_{SE}^{(B)}(B)=\ket{\psi_B(B)}\!\bra{\psi_B(B)}$.
We compute $\rho_S^{(A)}(B)=\mathrm{Tr}_E[\rho_{SE}^{(A)}(B)]$ explicitly; the same
result holds for $H_B$ either by direct repetition or by the general invariance
$\mathrm{Tr}_E[(\mathbb{I}\otimes V)\rho(\mathbb{I}\otimes V^\dagger)]
=\mathrm{Tr}_E[\rho]$.

Write $\ket{\psi_A(B)}$ in Schmidt-like grouped form by environment basis:
\begin{equation}
\ket{\psi_A(B)}=
\ket{0}_E\otimes \ket{v_0}_S+\ket{1}_E\otimes \ket{v_1}_S,
\end{equation}
where (reading off coefficients from \eqref{eq:psiA_B})
\begin{equation}
\ket{v_0}_S=\tfrac{1}{2}\big(e^{-i\theta}\ket{0}+e^{+i\theta}\ket{1}\big),
\qquad
\ket{v_1}_S=\tfrac{1}{2}\big(e^{+i\theta}\ket{0}+e^{-i\theta}\ket{1}\big).
\label{eq:v0v1}
\end{equation}
Then
\begin{equation}
\rho_S^{(A)}(B)=\mathrm{Tr}_E\!\big[\ket{\psi_A(B)}\!\bra{\psi_A(B)}\big]
=\ket{v_0}\!\bra{v_0}+\ket{v_1}\!\bra{v_1}.
\label{eq:rhoS_sum}
\end{equation}
Compute each term using \eqref{eq:v0v1}:
\begin{align}
\ket{v_0}\!\bra{v_0}
&=\tfrac{1}{4}\Big(\ket{0}\!\bra{0}+\ket{1}\!\bra{1}
+e^{-i2\theta}\ket{0}\!\bra{1}+e^{+i2\theta}\ket{1}\!\bra{0}\Big),
\\
\ket{v_1}\!\bra{v_1}
&=\tfrac{1}{4}\Big(\ket{0}\!\bra{0}+\ket{1}\!\bra{1}
+e^{+i2\theta}\ket{0}\!\bra{1}+e^{-i2\theta}\ket{1}\!\bra{0}\Big).
\end{align}
Summing,
\begin{align}
\rho_S^{(A)}(B)
&=\tfrac{1}{2}\Big(\ket{0}\!\bra{0}+\ket{1}\!\bra{1}\Big)
+\tfrac{1}{4}\Big((e^{-i2\theta}+e^{+i2\theta})\ket{0}\!\bra{1}
+(e^{+i2\theta}+e^{-i2\theta})\ket{1}\!\bra{0}\Big)
\nonumber\\
&=\tfrac{1}{2}\mathbb{I}
+\tfrac{1}{2}\cos(2\theta)\Big(\ket{0}\!\bra{1}+\ket{1}\!\bra{0}\Big)
\nonumber\\
&=\tfrac{1}{2}\Big(\mathbb{I}+\cos(2\theta)\,\sigma_x^{(S)}\Big).
\label{eq:rhoS_B_final}
\end{align}
As noted, $\rho_S^{(B)}(B)=\rho_S^{(A)}(B)$ by the invariance of the partial trace
under local unitaries on $E$.

\subsection{Post-boundary evolution and reduced diagnostic}

Define the post-boundary evolution block as applying the \emph{same} unitary
$U(\phi)$ to both arms:
\begin{equation}
\ket{\psi_{A,2}}=U(\phi)\ket{\psi_A(B)}, \qquad
\ket{\psi_{B,2}}=U(\phi)\ket{\psi_B(B)}.
\end{equation}

\paragraph{Arm $A$.}
Using \eqref{eq:psiA_B} and the action \eqref{eq:Uphi_action},
\begin{align}
\ket{\psi_{A,2}}
&=\tfrac{1}{2}\Big(e^{-i(\theta+\phi)}\ket{00}+e^{+i(\theta+\phi)}\ket{01}
+e^{+i(\theta+\phi)}\ket{10}+e^{-i(\theta+\phi)}\ket{11}\Big)
\nonumber\\
&=U(\theta+\phi)\ket{\psi_0}.
\label{eq:psiA2}
\end{align}
Therefore, by repeating the boundary calculation with $\theta\mapsto(\theta+\phi)$,
the reduced state on $S$ satisfies
\begin{equation}
\rho_S^{(A)}(\phi)=\tfrac{1}{2}\Big(\mathbb{I}+\cos\big(2(\theta+\phi)\big)\sigma_x^{(S)}\Big),
\end{equation}
so the diagnostic $D(\phi):=\langle\sigma_x^{(S)}\rangle$ is
\begin{equation}
D_A(\phi)=\mathrm{Tr}\big(\rho_S^{(A)}(\phi)\sigma_x^{(S)}\big)=\cos\big(2(\theta+\phi)\big).
\label{eq:DA_final}
\end{equation}

\paragraph{Arm $B$.}
Start from \eqref{eq:psiB_B}, then apply $U(\phi)$:
\begin{align}
\ket{\psi_{B,2}}
&=\tfrac{1}{2}\Big(e^{-i\phi}e^{+i\theta}\ket{00}+e^{+i\phi}e^{-i\theta}\ket{01}
+e^{+i\phi}e^{-i\theta}\ket{10}+e^{-i\phi}e^{+i\theta}\ket{11}\Big)
\nonumber\\
&=\tfrac{1}{2}\Big(e^{-i(\phi-\theta)}\ket{00}+e^{+i(\phi-\theta)}\ket{01}
+e^{+i(\phi-\theta)}\ket{10}+e^{-i(\phi-\theta)}\ket{11}\Big)
\nonumber\\
&=U(\phi-\theta)\ket{\psi_0}.
\label{eq:psiB2}
\end{align}
Thus, again repeating the reduced-state calculation with angle $(\phi-\theta)$,
\begin{equation}
\rho_S^{(B)}(\phi)=\tfrac{1}{2}\Big(\mathbb{I}+\cos\big(2(\phi-\theta)\big)\sigma_x^{(S)}\Big),
\end{equation}
and
\begin{equation}
D_B(\phi)=\cos\big(2(\phi-\theta)\big).
\label{eq:DB_final}
\end{equation}

\paragraph{Divergence under identical boundary state.}
Equations \eqref{eq:rhoS_B_final}, \eqref{eq:DA_final}, and \eqref{eq:DB_final}
show:
\begin{itemize}
\item At the conditioning boundary $B$, $\rho_S^{(A)}(B)=\rho_S^{(B)}(B)$ exactly.
\item Under identical post-boundary evolution $U(\phi)$, the reduced diagnostics differ
for generic $(\theta,\phi)$:
\begin{equation}
D_A(\phi)=\cos\big(2(\theta+\phi)\big)\neq \cos\big(2(\phi-\theta)\big)=D_B(\phi).
\end{equation}
\end{itemize}
This is the explicit algebraic realization of the Box~X pattern:
operationally indistinguishable boundary states admit distinct future extensions.

\subsection{Partial-trace invariance under environment relabelings}

For completeness, we record the general identity used above.
For any operator $\rho_{SE}$ and any unitary $V_E$ on $E$,
\begin{equation}
\mathrm{Tr}_E\!\big[(\mathbb{I}_S\otimes V_E)\,\rho_{SE}\,(\mathbb{I}_S\otimes V_E^\dagger)\big]
=\mathrm{Tr}_E[\rho_{SE}],
\end{equation}
because the partial trace over $E$ is basis-independent and conjugation by a local
unitary on $E$ amounts to a change of basis on the traced subsystem.

\subsection{Symmetry group $G$ and operational indistinguishability}
\label{subsec:symmetry-group}

We now make explicit the symmetry structure underlying the concrete example
and show why any state augmentation that restores sufficiency necessarily
breaks a physically meaningful symmetry.

\paragraph{Operational frame.}
Fix the operational frame $F_S$ in which, at the conditioning boundary $B$,
only measurements and controls acting on the system $S$ are available.
No measurements on the environment $E$, joint $SE$ observables, or records
of intermediate operations are accessible in this frame.

Within $F_S$, two preparations are operationally indistinguishable at $B$
iff they induce identical reduced states $\rho_S(B)$ and yield identical
statistics for all observables accessible in $F_S$.

\paragraph{Definition of the symmetry group $G$.}
Let $G$ denote the group of transformations acting on admissible histories
that satisfy the following two conditions:
\begin{enumerate}
\item \textbf{State invariance:} For any $g\in G$ and any history $H$,
the induced reduced boundary state is invariant,
\[
\rho_S(B; H) = \rho_S(B; g\cdot H).
\]
\item \textbf{Operational undetectability:}
No measurement or control available in frame $F_S$ can detect whether $g$ has been applied.
That is, for any observable $O_S$ accessible in $F_S$ and any history $H$,
\[
\langle O_S \rangle_{H}
=
\langle O_S \rangle_{g \cdot H}.
\]
\end{enumerate}

In the present example, $G$ consists of all local unitaries acting on $E$
at the boundary,
\[
G = \{\, \mathbb{I}_S \otimes V_E \mid V_E \in \mathrm{U}(2) \,\}.
\]
Each $g\in G$ acts nontrivially on admissible histories but trivially on
the reduced boundary description in $F_S$.

\paragraph{Relation between the histories $H_A$ and $H_B$.}
The two histories introduced in
Sec.~\ref{subsec:concrete-hamiltonian-example} are related by the action
of a specific element of $G$:
\begin{equation}
H_B = (\mathbb{I}_S \otimes \sigma_x^{(E)}) \circ H_A .
\end{equation}
Explicitly,
\begin{equation}
\ket{\psi_B(B)} =
(\mathbb{I}_S \otimes \sigma_x^{(E)}) \ket{\psi_A(B)},
\end{equation}
as shown in Eq.~\eqref{eq:psiB_B}.
By invariance of the partial trace,
\begin{equation}
\rho_S^{(B)}(B)
=
\mathrm{Tr}_E\!\left[
(\mathbb{I}_S \otimes \sigma_x^{(E)})
\rho_{SE}^{(A)}(B)
(\mathbb{I}_S \otimes \sigma_x^{(E)})^\dagger
\right]
=
\rho_S^{(A)}(B),
\end{equation}
establishing that $H_A$ and $H_B$ are $G$-equivalent and
indistinguishable at the boundary within $F_S$.

\paragraph{Operational indistinguishability at the boundary.}
Let $O_S$ be any observable acting on $S$ accessible in $F_S$.
Then
\begin{equation}
\langle O_S \rangle_A
=
\mathrm{Tr}\!\left[\rho_S^{(A)}(B)\,O_S\right]
=
\mathrm{Tr}\!\left[\rho_S^{(B)}(B)\,O_S\right]
=
\langle O_S \rangle_B .
\end{equation}
Thus \emph{no measurement available in frame $F_S$ can distinguish}
$H_A$ from $H_B$ at the conditioning boundary.
The symmetry $G$ therefore corresponds to experimentally verified
indistinguishability, not merely formal equivalence.

\paragraph{Failure of faithful state augmentation.}
Suppose one attempts to restore state sufficiency by augmenting the reduced
state with an auxiliary variable $\lambda$ indicating which history occurred,
\[
\rho_S(B) \longmapsto (\rho_S(B), \lambda),
\qquad \lambda \in \{A,B\}.
\]
This augmentation distinguishes
\[
(\rho_S(B), A) \neq (\rho_S(B), B),
\]
despite the fact that:
\begin{enumerate}
\item $H_A$ and $H_B$ are related by a $G$-transformation,
\item all $G$-transformations are operationally undetectable in $F_S$,
\item no boundary-accessible measurement can determine $\lambda$.
\end{enumerate}

The augmented variable $\lambda$ therefore encodes distinctions that have
no operational support within the frame in which the reduced state is defined.
In the sense of Definition~\ref{def:faithful-state-representation},
this augmentation violates symmetry preservation and is unfaithful.

The divergence of diagnostics despite boundary-state equality
(Eqs.~\eqref{eq:diagnostic_divergence})
then follows as an operational consequence: future behavior depends on information
(the $G$-equivalence class of the history) that is not encoded in, and cannot be
recovered from, the reduced state $\rho_S(B)$ within frame $F_S$.

\paragraph{Physical meaning of symmetry breaking.}
To empirically verify the auxiliary variable $\lambda$ \emph{within frame $F_S$},
one would need to:
\begin{enumerate}
\item measure observables on $E$ (outside $F_S$ by definition),
\item perform joint $SE$ tomography (which requires access to $E$), or
\item retain records of the temporal ordering of operations during preparation
(requiring tracking of operations outside $F_S$).
\end{enumerate}

\textbf{Crucially, all of these operations are unavailable in frame $F_S$ by construction.}
Any measurement capable of verifying $\lambda$ would therefore constitute an
\emph{enlargement of the operational frame}, under which the $G$-equivalence
established at the boundary no longer holds.

Thus, postulating $\lambda$ as part of the state description within $F_S$ amounts
to claiming knowledge of information that is:
\begin{itemize}
\item not encoded in any $F_S$-accessible observable,
\item not verifiable by any measurement available at the boundary, and
\item only meaningful in an enlarged frame where $G$-symmetry is explicitly broken.
\end{itemize}

This is the precise sense in which state augmentation restores prediction only by
abandoning faithful representation: it imports distinctions from outside the
representational frame while claiming they belong inside it.

\paragraph{Analogy to gauge freedom.}
The situation is analogous to gauge freedom in electromagnetism:
different gauge choices yield identical physical predictions, yet any attempt to
promote gauge choice itself to a physical state variable breaks gauge invariance.
Here, $G$-equivalence classes play the role of gauge orbits, while the auxiliary
variable $\lambda$ functions as a gauge-fixing parameter that has no physical
meaning within the gauge-invariant (operationally reduced) description.

\paragraph{Optional frame refinement.}
If the operational frame is enlarged to $F_{SE}$, granting access to measurements
on $E$ at the boundary, the symmetry group contracts to the identity,
$G \to \{\mathbb{I}\}$, and state sufficiency is restored at the level of
$\rho_{SE}(B)$. The appearance of $\Sigma_2$ structure is therefore
frame-relative but non-arbitrary: it reflects objective operational
indistinguishability within the chosen frame.

\subsection{Remark on operational meaning}

In an operational frame $F_S$ where only $S$ is accessible at the boundary, the
two preparations are indistinguishable at $B$ because they induce identical
$\rho_S(B)$ and any boundary-accessible measurement is a function of $\rho_S(B)$.
Nevertheless, the fixed post-boundary interaction with $E$ makes the subsequent
reduced behavior depend on history-level information not recoverable from
$\rho_S(B)$ alone. This is precisely the representational regime that motivates
$\Sigma_2$ diagnostics.