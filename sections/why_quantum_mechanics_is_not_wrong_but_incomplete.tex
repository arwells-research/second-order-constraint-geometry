\section{Why Quantum Mechanics Is Not Wrong—but Structurally Limited}
\label{sec:qm-not-wrong-but-incomplete}

Quantum mechanics is among the most empirically successful frameworks in the
history of science. Its predictive accuracy, internal consistency, and
experimental confirmation are not in question. Nothing in the present work
challenges its formal structure, its dynamical laws, or its statistical
predictions.

The claim advanced here is narrower and structural. Quantum mechanics provides an
exact and complete account of states and their evolution, but it requires
supplementary operational structure in regimes where admissibility depends on
ordered history rather than instantaneous configuration alone. This requirement
does not reflect missing dynamics, incorrect equations, or empirical failure. It
reflects a representational limit of state-based description.

The quantum postulates assert that the physical state of a system at an instant of
time, together with the Hamiltonian, suffices to determine all future measurement
statistics. This first-order sufficiency claim is correct within the formalism and
adequate for a wide range of phenomena. However, it does not exhaust the question
of admissibility. In situations where future realizability depends on the
ordering of prior transformations, on preparation history, or on conditioning
procedures, the instantaneous quantum state remains predictive but does not
uniquely determine which future operations remain physically admissible.

In such cases, the quantum state is not incorrect. It is operationally
insufficient for encoding admissibility. Distinct histories may give rise to the
same reduced state while differing in which extensions, reversals, or control
operations remain realizable. This distinction is invisible at the level of
state description but decisive at the level of operational behavior.

Crucially, this limitation is not an interpretational issue. Competing
interpretations of quantum mechanics—collapse versus no collapse, epistemic
versus ontic states, many worlds versus hidden variables—agree on the same
predictions wherever state sufficiency holds. The phenomena motivating
$\Sigma_2$ arise when admissibility itself depends on history, even under fixed
state description. The issue therefore precedes any metaphysical interpretation
of the quantum state.

From this perspective, the persistent sense that quantum mechanics resists
intuition has a specific structural origin. The theory is extraordinarily precise
about states, amplitudes, and statistics, but largely silent about certain
ordering-dependent constraints that govern physical organization. These
constraints are operationally real and experimentally accessible, yet they are
not representable as properties of instantaneous states. Their absence from the
formalism reflects a boundary of representational scope, not a failure of the
theory.

This position does not conflict with existing no-go theorems. Results such as
Bell’s theorem, the Kochen--Specker theorem, and the no-cloning theorem restrict
state-based completions of quantum mechanics. Second-order constraint geometry
does not attempt such a completion. It introduces no hidden variables, modifies no
dynamics, and preserves all established impossibility results. Instead, it
clarifies when successful quantum descriptions rely on admissibility constraints
that are not encoded at the level of state alone.

Describing quantum mechanics as structurally limited in this sense is therefore
not a criticism. It is a statement about scope. Quantum mechanics is complete as a
theory of states and their evolution. It does not, in general, provide a complete
representation of admissible histories in regimes where ordering-sensitive
constraints matter. Recognizing this distinction explains why repeated attempts
to enforce universal state sufficiency encounter structural barriers rather than
empirical ones.

Second-order constraint geometry does not compete with quantum mechanics. It
articulates the boundary at which state-based description reaches its natural
limit and provides a minimal, non-dynamical language for expressing admissibility
structure that quantum theory already relies upon operationally but does not
itself encode.