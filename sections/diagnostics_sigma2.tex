\section{Diagnostics for Second-Order Constraints}
\label{sec:diagnostics-sigma2}

The definition of second-order constraint geometry $\Sigma_2$ in
Sec.~\ref{sec:definition-sigma2} is purely structural. The present section
establishes how such constraints may be diagnosed operationally, without assuming
any particular physical mechanism or domain-specific realization.

The guiding principle mirrors that of projection-induced arrow diagnostics: when a
constraint acts at the level of histories rather than instantaneous states, any
reduction that erases history-level relations must exhibit characteristic and
unavoidable failures. These failures are not artifacts of poor modeling; they are
diagnostic consequences of attempting to represent trajectory-level structure in
a state-based language.

\subsection{Operational reduction and diagnostic collapse}

Let $\Pi : \mathcal{H} \to \mathcal{D}$ denote an operational reduction mapping
admissible histories to a reduced description space $\mathcal{D}$.

\begin{definition}[Faithful reduction]
An operational reduction $\Pi$ is faithful with respect to a constraint $C$ on
$\mathcal{H}$ if distinct admissibility classes induced by $C$ remain
distinguishable in $\mathcal{D}$.
\end{definition}

If $\Pi$ is not faithful to a second-order constraint, then histories that differ
only by satisfaction or violation of that constraint are necessarily mapped to the
same reduced description. The loss is not merely informational: distinctions that
govern admissibility in $\mathcal{H}$ become inexpressible in $\mathcal{D}$.

\subsection{The diagnostic principle}

This observation leads directly to a general diagnostic result.

\begin{proposition}[Second-order diagnostic principle]
\label{prop:sigma2-diagnostic}
If an admissible-history space $\mathcal{H}$ is subject to a second-order
constraint $C \in \Sigma_2$, then any operational reduction
$\Pi : \mathcal{H} \to \mathcal{D}$ that discards history-level relations must
exhibit at least one of the following features: irreducible degeneracy between
operationally distinct preparations, apparent stochasticity or irreversibility not
attributable to first-order dynamics, or dependence on implicit boundary conditions
or ordering conventions.
\end{proposition}

\begin{proof}[Proof sketch]
Because $C$ cannot be factorized into state-level predicates, its satisfaction
depends on relations between distinct segments of a history. Any reduction $\Pi$
that represents only instantaneous or local state information necessarily
identifies histories that differ only by $C$. Operational distinctions that track
admissibility in $\mathcal{H}$ therefore collapse in $\mathcal{D}$, forcing at least
one of the listed diagnostic manifestations.
\end{proof}

The proposition does not enumerate independent symptoms. It describes different
faces of the same structural failure: the attempt to encode trajectory-dependent
admissibility using state-local objects.

\subsection{Boundary dependence and admissibility loss}

One common manifestation of diagnostic collapse is the introduction of explicit or
implicit boundary conditions. When admissibility depends on relations between
history segments but the reduced description lacks access to those relations, the
only remaining means of enforcing consistency is through privileged boundaries:
initialization procedures, measurement cuts, reset conventions, or conditioning
events.

Boundary dependence here does not assert temporal asymmetry. It refers to any
operational procedure that must fix a reference segment of a history in order to
remain predictive. The appearance of such boundaries signals that admissibility is
no longer encoded locally in state.

This diagnostic overlaps structurally with projection-induced arrows, but the
origin differs. In the present case, the asymmetry arises not from projection
itself, but from the attempt to represent history-level constraints within a
state-based framework.

\subsection{Order sensitivity and non-commutativity}

A second, closely related diagnostic arises when admissible transformations fail to
commute under composition.

\begin{definition}[Order sensitivity]
A system exhibits order sensitivity if two admissible transformations
$T_1, T_2$ satisfy
\[
T_2 \circ T_1 (h) \in \mathcal{H}
\quad \text{but} \quad
T_1 \circ T_2 (h) \notin \mathcal{H}.
\]
\end{definition}

This non-commutativity is not algebraic but admissibility-based. Both compositions
may be well-defined as sequences of operations, yet only one preserves membership
in $\mathcal{H}$. Such behavior cannot be reduced to instantaneous state variables
without explicitly encoding ordering or history information. It therefore provides
a direct diagnostic of second-order constraint structure.

\subsection{Distinction from hidden-state explanations}

It is essential to distinguish $\Sigma_2$ diagnostics from hidden-variable or
memory-based explanations. A hidden-state model restores state locality by
augmenting the state space $\mathcal{S}$ so that admissibility becomes
instantaneous again.

A genuine second-order constraint does not disappear under such augmentation unless
the augmentation breaks the symmetries of the admissible-history space. The
diagnostic question is therefore not whether a state-augmented model can reproduce
observations, but whether it can do so while preserving the original admissibility
structure without introducing privileged parametrizations, bookkeeping variables,
or boundary asymmetries.

\subsection{Falsifiability}

The $\Sigma_2$ framework is falsifiable in the same sense as other structural
constraints. If an apparent second-order diagnostic can be eliminated by a
symmetry-preserving, state-local reformulation—without introducing boundary
dependence, ordering conventions, or hidden asymmetries—then the invocation of
$\Sigma_2$ is unwarranted.

Conversely, the persistence of diagnostic signatures under all faithful reductions
constitutes evidence that admissibility is not fully state-local. In such cases,
history-level constraint structure is not an interpretive choice but an operational
necessity.

\subsection{Scope}

This section establishes diagnostic criteria only. It does not assert that any
particular physical system instantiates second-order constraints. The role of
$\Sigma_2$ is to make explicit when history-level structure is being implicitly
relied upon, and to provide a principled language for assessing whether such
structure is fundamental or merely representational.

Concrete applications and empirical case studies are taken up in subsequent
sections.