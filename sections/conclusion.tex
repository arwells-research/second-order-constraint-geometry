\section{Conclusion}
\label{sec:conclusion}

This paper has shown that state sufficiency is not a universal representational
regime. There exist admissible dynamics for which future behavior depends on
ordering and trajectory constraints that cannot be recovered from instantaneous
state descriptions alone. When such dependence is present, a state-based geometry
is structurally incomplete as a faithful representation..

Within the state-sufficient regime, standard quantum mechanics remains correct and
operationally complete. Outside that regime, no modification of dynamics is
required. What is required is an additional geometric layer capable of expressing
constraints that act on admissible histories rather than on instantaneous states.
Second-order constraint geometry, $\Sigma_2$, provides this minimal structure.

The empirical content of the framework lies in the separation it enforces. By
holding reduced state statistics fixed while varying admissible history, the
protocols introduced here establish a sharp, falsifiable boundary between regimes
in which state sufficiency holds and regimes in which it fails. In the latter
case, history dependence is not an interpretive choice but an operational fact.

Together with the companion result on projection-induced arrows—which explains directional asymmetries under symmetric dynamics—this work clarifies exactly when and why state-based descriptions reach their limits. Arrows of directedness arise from projection; failures of state sufficiency arise from trajectory-level constraint structure. Both follow inevitably once admissible histories are taken as primary.

Second-order constraint geometry is therefore not an alternative theory, but the
next necessary layer of description when admissibility depends on history rather
than on state. Open questions include characterizing the full class of physically realizable $\Sigma_2$ structures, developing computational methods for identifying second-order constraints in complex systems, and exploring whether $\Sigma_2$ admits natural generalizations to higher-order trajectory dependence. The contribution is classificatory rather than predictive: it identifies when new predictions are impossible without expanding the representational frame.