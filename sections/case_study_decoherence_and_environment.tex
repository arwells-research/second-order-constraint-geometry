\section{Case Study: Decoherence and Environmental Coupling}
\label{sec:case-decoherence-environment}

Decoherence provides a second canonical example in which instantaneous state
descriptions are insufficient to characterize admissible physical behavior.
Unlike measurement, decoherence involves no explicit projection postulate and
is often presented as fully reducible to unitary dynamics.
This makes it a particularly clean test case for the necessity of
second-order constraint structure.

\subsection{Standard decoherence setup}

Consider a system $S$ interacting unitarily with an environment $E$.
The joint state evolves according to
\[
\rho_{SE}(t) = U_{SE}(t)\,\rho_{SE}(0)\,U_{SE}^\dagger(t),
\]
with $U_{SE}$ generated by a time-reversal symmetric Hamiltonian.

Decoherence analyses typically focus on the reduced state
\[
\rho_S(t) = \mathrm{Tr}_E[\rho_{SE}(t)],
\]
whose off-diagonal terms in a preferred basis decay rapidly under generic
system–environment couplings.

\subsection{State-based description and its limits}

At any given time $t$, the reduced density matrix $\rho_S(t)$ provides a
complete description of all future measurement statistics on $S$ alone.
In this sense, decoherence introduces no deviation from standard quantum
mechanics.

However, the reduced state $\rho_S(t)$ does not encode the admissibility of
future operations that involve the environment.

In particular, two global histories can yield identical reduced states while
differing in whether recoherence, interference revival, or reversal operations
are physically realizable.

\subsection{History dependence of admissibility}

Consider two scenarios producing the same $\rho_S(t)$:

\begin{enumerate}
\item The system becomes entangled with a small, controllable environment,
      such that the joint evolution is, in principle, reversible.
\item The system becomes entangled with a large, uncontrolled environment,
      such that phase information is effectively dispersed into inaccessible
      degrees of freedom.
\end{enumerate}

At the level of $\rho_S(t)$, these scenarios are indistinguishable.
Yet the admissibility of subsequent operations—such as recoherence protocols,
Loschmidt echoes, or partial reversal—differs dramatically between them.

This distinction cannot be recovered from the instantaneous reduced state.

\subsection{Second-order constraint interpretation}

Within the $\Sigma_2$ framework, decoherence is understood as the imposition of
constraints on admissible histories rather than as a change in fundamental
dynamics.

Environmental coupling restricts the set of continuations compatible with a
given reduced state.
As entanglement spreads into uncontrolled degrees of freedom, the space of
admissible trajectories contracts.

Crucially, this contraction is path-dependent: it depends on how the system
arrived at its present reduced state, not merely on the state itself.

The decoherence and recoherence scenarios discussed in this section similarly
instantiate the symmetry-based diagnostic introduced in
Sec.~\ref{subsec:spin-half-symmetry-example}. Histories that yield the same reduced
state $\rho_S(t)$ may nonetheless differ in whether recoherence or reversal
operations are admissible, depending on correlations with environmental degrees
of freedom that are operationally inaccessible. Transformations acting on these
inaccessible degrees of freedom define an admissible-history symmetry: they leave
the reduced state invariant while acting nontrivially on the space of admissible
future continuations.

Viewed in this way, the insufficiency of the reduced state is not simply a
reminder that questions about joint system--environment operations require a
larger Hilbert space. Rather, it reflects the same structural situation identified
in Sec.~3.5: when admissibility depends on history-level relations preserved under
a physically motivated symmetry, no symmetry-preserving state representation can
encode all admissibility information. Decoherence thus provides a concrete
context in which second-order constraints arise as a representational issue,
independently of predictive success.

\subsection{Relation to arrow-like behavior}

Decoherence is often cited as explaining the emergence of classicality and
temporal asymmetry.
From the present perspective, this asymmetry arises from conditioning on
environmental correlations rather than from time-asymmetric laws.

This mirrors the structure identified in projection-induced arrow analyses:
the reduced description is anchored to a conditioning boundary that restricts
admissible histories in one direction.
The resulting arrow-like behavior is therefore boundary-relative, not intrinsic.

\subsection{Why decoherence does not eliminate $\Sigma_2$}

Decoherence is sometimes invoked as rendering higher-order structure unnecessary.
However, the symmetry-based diagnostic of Sec.~3.5 shows the opposite: decoherence makes second-order constraints unavoidable.

Even when all dynamics are unitary, the admissibility of operations depends on
ordered history, environmental accessibility, and prior entanglement structure.
These features are not representable within a purely state-based geometry.

\subsection{Summary}

Decoherence provides a paradigmatic case in which:
\begin{enumerate}
\item identical reduced states correspond to distinct admissible futures,
\item admissibility depends on environmental history and accessibility,
\item state descriptions suffice for prediction but not for operational reach.
\end{enumerate}

This establishes decoherence as a natural domain for second-order constraint
geometry and reinforces the general claim that $\Sigma_2$ captures structure
already implicit in quantum practice.