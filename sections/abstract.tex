\begin{abstract}
Many physical and computational systems exhibit behavior that depends not only on
their instantaneous state, but on the ordering, admissibility, or history of prior
transformations. Standard state-based formalisms accommodate such behavior only by augmenting the state with hidden registers, contextual labels, or implicit memory.
This effectively blurs the distinction between genuine physical structure and bookkeeping artifacts.

In this work we introduce \emph{second-order constraint geometry} ($\Sigma_2$), a
geometric framework defined over admissible trajectories rather than instantaneous
states. $\Sigma_2$ formalizes when and why state descriptions are insufficient, without
appeal to hidden variables or enlarged state spaces. The framework characterizes
order-sensitive and history-dependent phenomena as arising from constraints on
admissible sequences of transformations, rather than from additional state content.

We show that $\Sigma_2$ provides a principled criterion distinguishing legitimate
state evolution from illegitimate state augmentation, clarifying long-standing
confusions surrounding measurement, post-selection, decoherence, and contextuality in
quantum theory. The framework is non-mechanistic and non-dynamical by construction: it introduces no new
equations of motion, parameters, or stochastic processes. Instead, it functions as a
diagnostic and classificatory layer, constraining how explanations may be structured.

Concrete toy models, operational diagnostics, and experimentally testable protocols
are presented to demonstrate falsifiability. We further situate $\Sigma_2$ in relation
to projection-induced arrow constraints, showing how second-order organization
complements geometric accounts of irreversibility while remaining orthogonal to boundary selection and dynamical asymmetry. A fully worked two-qubit Hamiltonian example establishes physical realizability within standard unitary quantum mechanics, and an operational protocol supplies quantitative falsification criteria. The result is a minimal, falsifiable framework for understanding the limits of state sufficiency across physics, computation, and complex systems.
\end{abstract}