\section{Relation to Projection-Induced Arrows}
\label{sec:relation-projection-arrows}

The projection-induced arrow theorem and the second-order constraint geometry
$\Sigma_2$ address the same structural limitation from different descriptive
levels. The former establishes when arrow-like diagnostics are unavoidable under
operational reduction, while the latter identifies the minimal geometric structure
required once admissibility depends on history rather than instantaneous state.
This section makes their relationship explicit.

\subsection{Two views of the same limitation}

The projection-induced arrow result establishes a no-go constraint: under symmetric
admissible dynamics and boundary-symmetric operational reduction, no intrinsic
direction-sensitive diagnostic can arise. Any observed arrow must therefore enter
through asymmetry in dynamics, asymmetry in projection, or selection of a
conditioning boundary.

Second-order constraint geometry addresses the complementary question that follows
once this constraint is acknowledged. When admissibility depends on ordered
history rather than on instantaneous configuration, what structure is required to
represent that dependence without introducing spurious asymmetry?

In both frameworks, the answer is the same. Arrow-like behavior appears when future
realizability is constrained relative to a conditioning boundary in a manner that
cannot be represented at the level of instantaneous state.

\subsection{Projection as a special case of admissibility loss}

Operational projection $\Pi : \mathcal{H} \to \mathcal{D}$ discards relations among
histories that are relevant to admissibility. When the projection is
boundary-symmetric, this loss removes orientation information, forcing reduced
diagnostics to depend only on unsigned separation from the conditioning boundary.

This is a specific instance of a more general phenomenon. The reduced description
fails to encode which future histories remain admissible, not merely which states
are occupied. The loss is therefore operational rather than informational.

In the language of $\Sigma_2$, projection collapses distinct admissible-history
classes—each with different future-accessibility relations—into a single reduced
state. Arrow-like diagnostics then arise as a consequence of this collapse.

\subsection{Arrows as second-order constraints}

From the perspective of second-order constraint geometry, arrow-like diagnostics do
not measure intrinsic temporal direction. They quantify separation in admissibility
space from a conditioning boundary. Their magnitude reflects the accumulation of
constraints on which histories remain physically realizable.

This classification explains why arrow-like behavior is compatible with
time-reversal symmetric dynamics, insensitive to microscopic reversibility, and
dependent on preparation, conditioning, or environmental coupling. In each case,
the arrow reflects restrictions on admissible continuations rather than properties
of the equations governing evolution along a history.

\subsection{Boundary selection and admissibility}

The projection-induced arrow theorem deliberately remains agnostic about why a
particular conditioning boundary is physically realized. Second-order constraint
geometry sharpens this distinction by separating the role of boundary selection
from the structure of admissibility itself.

Boundary selection determines which admissibility constraints apply. $\Sigma_2$
determines how those constraints restrict future realizability once a boundary is
fixed. Maintaining this separation prevents a common category error: treating
boundary-relative directedness as evidence for intrinsic temporal asymmetry in the
dynamics.

\subsection{Unified structural interpretation}

Taken together, the two results yield a unified structural picture. Symmetric
admissible dynamics define the full space of possible histories. Operational
reductions identify histories that are distinct at the level of admissibility.
Boundary selection restricts which histories are physically realized. Second-order
constraint geometry encodes the resulting restrictions on future realizability,
and arrow-like diagnostics measure separation from the boundary within this
constrained admissibility structure.

In this sense, the projection-induced arrow theorem identifies the necessary
conditions under which arrows must appear, while $\Sigma_2$ provides the minimal
geometric language required to describe them once state sufficiency fails.

The recurrence of arrow-like phenomena across thermodynamics, decoherence, and
computation is therefore not a dynamical mystery. It is a structural consequence of
operating within reduced descriptions when admissibility depends on history rather
than on state.