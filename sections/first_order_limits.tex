\section{Limits of First-Order (State-Based) Descriptions}
\label{sec:first-order-limits}

Section~\ref{sec:limits-state-sufficiency} established the failure of state
sufficiency at the operational level: identical instantaneous descriptions can
permit distinct admissible futures. This section formalizes the failure by identifying constraints that first-order descriptions cannot encode without structural distortion. This insufficiency holds relative to any fixed operational frame where admissibility and symmetry are defined.

Many physical, computational, and informational theories represent system behavior
using a \emph{first-order} description: the system is assumed to be fully
characterized at any instant by a state variable, with future evolution determined
by a rule acting on that state. This structure underwrites much of classical
mechanics, statistical mechanics, stochastic processes, and quantum theory.

The failure is not empirical or approximate, but geometric:
it arises when admissibility depends on relations between segments of a history
rather than on instantaneous state alone.

\subsection{First-order sufficiency}

Let $\mathcal{S}$ denote a state space, and let admissible evolution be specified by
a map
\[
s_{t+1} = F(s_t),
\]
or more generally by a transition kernel $P(s_{t+1} \mid s_t)$. A description of
this form is \emph{first-order sufficient} if the instantaneous state encodes all
information relevant to future admissibility and observable behavior, such that
conditioning on $s_t$ renders future evolution independent of prior history.

When apparent history dependence arises, the standard response is to enlarge the
state space so that the augmented state restores Markovian closure without altering
the structural form of the dynamics. This move is often treated as unproblematic,
and indeed succeeds whenever the relevant dependence can be represented as
additional instantaneous variables.

These assumptions jointly define what it means for a description to be genuinely
first-order: admissibility factorizes through state, and any required refinement
can be absorbed into a larger state space.

\subsection{Histories and admissibility}

Let $\mathcal{H}$ denote the space of admissible histories, where each history
$h \in \mathcal{H}$ is an ordered sequence (or continuous curve) of states consistent
with the system’s underlying laws or constraints. A first-order description
implicitly assumes that admissibility of a history can be factorized into
admissibility of its instantaneous states together with the local transition rule
$F$.

This factorization fails when admissibility depends on relations between distinct
segments of a history rather than on instantaneous values alone. In such cases,
ordering of transformations, feasibility of entire sequences, or compatibility
with boundary conditions determines which continuations remain physically
realizable. Admissibility is then a property of the history as a whole—*within a
specified operational frame*- not of its individual states.

\subsection{Why state augmentation fails}

A natural attempt to restore first-order sufficiency in such situations is to
enlarge the state space to encode relevant historical information. This strategy
succeeds only when the constraint of interest is representational—when it can be
captured as an additional instantaneous property without altering the symmetry
structure of the admissible-history space.

However, when admissibility depends on equivalence classes of histories under
transformations that leave instantaneous states invariant, no symmetry-preserving state augmentation can encode the relevant structure
without explicitly breaking those invariances \emph{at the representational level}. Any such encoding necessarily selects a representative from each equivalence class.
This imports privileged parameters or boundary choices that are neither present nor independently verifiable within the original operational frame.

The resulting description may reproduce observed behavior, but it no longer
preserves the symmetries of the admissible-history space it purports to describe.
The augmentation resolves prediction at the cost of structural fidelity.

\subsection{Structural insufficiency of first-order descriptions}

It is therefore essential to distinguish between \emph{dynamical insufficiency},
which can be remedied by refining equations or adding variables, and
\emph{structural insufficiency}, which cannot.

\begin{proposition}[Structural insufficiency of first-order descriptions]
\label{prop:first-order-insufficiency}
There exist admissible-history spaces $\mathcal{H}$ equipped with symmetry-preserving
admissibility constraints for which no state-based representation can simultaneously
determine admissible future continuations from instantaneous state and preserve the
symmetry structure of admissible-history equivalence classes.

This is an existence claim: it does not assert that all physical systems exhibit
such structure, only that such systems are physically realizable within standard
quantum dynamics.

In such cases, any state augmentation that restores predictive completeness does so
only by introducing representational asymmetry, privileged boundary structure, or
explicit history bookkeeping not present in the original admissible-history space.
\end{proposition}

\begin{proof}[Proof sketch]
Let $\mathcal{H}$ be an admissible-history space equipped with a symmetry group $G$
acting on histories, inducing an equivalence relation $h \sim h'$ whenever $h$ and
$h'$ are related by an element of $G$. Assume that the action of $G$ preserves all
instantaneous state projections, so that histories equivalent under $G$ are
indistinguishable at every individual time slice.

Let $C$ be an admissibility constraint on $\mathcal{H}$ whose satisfaction depends on
relational properties between distinct segments of a history and is invariant under
the action of $G$. Such a constraint partitions $\mathcal{H}$ into admissibility
classes that are not determined by instantaneous state alone.

The relevant symmetry group $G$ is taken to represent experimentally verified
indistinguishability relations within the operational frame under consideration,
not abstract or post hoc equivalences.

Suppose, for contradiction, that there exists a state-based representation
$\Pi : \mathcal{H} \to \mathcal{S}$ that determines admissible future continuations
solely from $\Pi(h)$ while preserving the equivalence classes induced by $G$.
Because $C$ is not factorizable into instantaneous predicates, predictive sufficiency
requires $\Pi$ to distinguish between histories that differ only in their relational
structure. However, such histories lie in the same equivalence class under $G$.
Distinguishing them therefore violates $G$-invariance by introducing implicit ordering,
boundary marking, or trajectory bookkeeping.

Thus no state-based representation can simultaneously determine admissible future
evolution and preserve admissible-history symmetry. The insufficiency is therefore
structural rather than epistemic.
\end{proof}

This result is not definitional: it follows from the requirement that any
representation preserve the symmetry structure of admissible-history
equivalence classes that are operationally verified within the chosen frame.
Section~\ref{subsec:concrete-hamiltonian-example} provides an explicit physical
realization of this situation.

To distinguish genuine representational sufficiency from surrogate encodings that
relocate history dependence, we introduce the following notion of faithfulness.

A fully worked example with explicit Hamiltonians, demonstrating that histories
$H_A$ and $H_B$ related by operationally undetectable transformations yield
identical boundary states yet divergent diagnostics, is given in
Appendix~\ref{subsec:symmetry-group}. The example shows explicitly why any state
augmentation distinguishing admissible futures necessarily violates symmetries
corresponding to experimentally verified indistinguishabilities.

\subsection{Worked example: symmetry-induced failure of state sufficiency}
\label{subsec:spin-half-symmetry-example}

The abstract symmetry argument in Proposition~\ref{prop:first-order-insufficiency}
can be made fully explicit with a minimal physical example in which the relevant
symmetry group is operationally motivated rather than chosen ad hoc.

\paragraph{System and admissible histories.}
Consider a spin-$\tfrac{1}{2}$ system $S$ interacting with a measurement apparatus
$A$. We assume standard quantum mechanics throughout. Admissible histories
$h \in \mathcal{H}$ consist of ordered sequences of preparation, interaction, and
registration events, possibly including intermediate couplings to uncontrolled
degrees of freedom of $A$. No assumptions are made about collapse or interpretation;
only operational distinguishability is relevant.

We restrict attention to histories that yield the same verified reduced state
$\rho_S$ for $S$ at a designated reference time $t^\ast$, after tracing over
apparatus degrees of freedom. By construction, all such histories are
indistinguishable at the level of instantaneous state.

\paragraph{Operational symmetry group.}
Let $G$ be the group of relabelings of inaccessible apparatus degrees of freedom
that leave all reduced statistics of $S$ invariant. Concretely, elements of $G$
act by unitary transformations on $A$ that commute with all observables accessible
to the experimenter and therefore have no observable effect on $\rho_S$ at any
time slice.

This symmetry is not imposed for mathematical convenience; it reflects a physical
indistinguishability. Different microscopic apparatus configurations related by
$G$ cannot be operationally resolved and therefore represent the same physical
situation at the level of instantaneous state.

\paragraph{Trivial action on instantaneous state.}
For any two histories $h,h' \in \mathcal{H}$ related by $g \in G$, the reduced
state projections satisfy
\[
\Pi(h)(t) = \Pi(h')(t)
\quad \text{for all } t,
\]
where $\Pi$ denotes projection onto the reduced state of $S$. Thus $G$ acts
trivially on instantaneous state descriptions.

\paragraph{Nontrivial action on histories.}
Despite this state-level equivalence, histories related by $G$ may differ in
their admissible future continuations. For example, correlations between $S$
and inaccessible apparatus degrees of freedom may or may not be reversibly
recoverable depending on the prior interaction structure, even when the reduced
state $\rho_S$ is identical.

Operationally, this difference appears when considering admissibility of future
extensions: reversal protocols, recoherence operations, or consistent embedding
into a larger Hilbert space may be possible for some histories and not others.
These distinctions depend on relational properties across the history and are
invariant under $G$.

\paragraph{Failure of faithful state sufficiency.}
Because admissible future continuations differ, we have
\[
\mathcal{C}^+(h) \neq \mathcal{C}^+(h')
\]
for some $G$-equivalent histories satisfying $\Pi(h)=\Pi(h')$. Any state-based
representation that determines admissibility from instantaneous state therefore
fails admissibility sufficiency.

Attempting to restore sufficiency by augmenting the state would require encoding
which correlations with apparatus degrees of freedom remain recoverable. However,
this necessarily distinguishes between histories related by $G$, violating symmetry
preservation. The augmentation succeeds only by selecting privileged representatives
from each equivalence class of histories.

\paragraph{Interpretation.}
This example shows that the symmetry group $G$ arises from physical
indistinguishability, not from representational choice. Its action is trivial on
instantaneous state but nontrivial on admissible histories. The resulting failure
of faithful state sufficiency is therefore structural, not interpretational, and
cannot be eliminated by symmetry-preserving state augmentation.

The example supplies a concrete instance of Proposition~\ref{prop:first-order-insufficiency}
and motivates the faithfulness criteria introduced below.

\begin{definition}[Faithful state representation]
\label{def:faithful-state-representation}
A state-based representation of an admissible-history space $\mathcal{H}$ is
\emph{faithful} if there exists a projection
\[
\Pi : \mathcal{H} \to \mathcal{S}
\]
to a state space $\mathcal{S}$ such that:

\begin{enumerate}
\item \textbf{Admissibility sufficiency:}
For any two histories $h_1,h_2 \in \mathcal{H}$,
\[
\Pi(h_1) = \Pi(h_2)
\;\Rightarrow\;
\mathcal{C}^+(h_1) = \mathcal{C}^+(h_2),
\]
where $\mathcal{C}^+(h)$ denotes the set of admissible future continuations of $h$.

\item \textbf{Symmetry preservation:}
If $h_1 \sim h_2$ under an admissible-history equivalence relation generated by a
symmetry group $G$, then
\[
\Pi(h_1) = \Pi(h_2).
\]
That is, the state representation does not distinguish between histories that are
equivalent under the symmetries of $\mathcal{H}$.

\item \textbf{Boundary neutrality:}
The projection $\Pi$ does not require the introduction of privileged temporal
orientations, boundary selections, or ordering conventions that are not invariant
under the symmetries of $\mathcal{H}$.
\end{enumerate}

If no such projection exists, the admissibility structure of $\mathcal{H}$ is
said to be \emph{state-insufficient}.
\end{definition}

\subsection{Concrete Hamiltonian example: identical boundary state, distinct admissible futures}
\label{subsec:concrete-hamiltonian-example}

We now give a fully specified, physically realizable instance in which two
admissible histories yield the \emph{same verified reduced boundary state} for the
system, yet produce \emph{distinct post-boundary diagnostics} under identical
subsequent evolution. The point is not that global unitary quantum mechanics
fails (it does not), but that \emph{within a fixed operational frame} in which the
environment is inaccessible at the boundary, admissibility of future operations
fails to factorize through the reduced state.

\paragraph{System, environment, and Hamiltonian.}
Let $S$ and $E$ be qubits with joint Hilbert space
$\mathcal{H}_{SE} = \mathbb{C}^2 \otimes \mathbb{C}^2$.
Assume the post-boundary evolution block is generated by the fixed,
time-reversal-symmetric interaction Hamiltonian
\begin{equation}
H_{SE} \;=\; \hbar g\, \sigma_z^{(S)} \otimes \sigma_z^{(E)} ,
\label{eq:H_ZZ}
\end{equation}
so that the joint unitary for interaction time $t$ is
\begin{equation}
U(t) \;=\; \exp\!\big(-\tfrac{i}{\hbar} H_{SE} t\big)
\;=\; \exp\!\big(- i \phi\, \sigma_z^{(S)} \otimes \sigma_z^{(E)}\big),
\qquad \phi := g t .
\label{eq:U_ZZ}
\end{equation}

\paragraph{Operational frame.}
Fix an operational frame $F_S$ in which only measurements and controls on $S$ are
available at the conditioning boundary $B$ (the environment $E$ is not accessible
there). In this frame, two global preparations are \emph{operationally
indistinguishable at $B$} iff they induce the same reduced state $\rho_S(B)$ and
no available boundary procedure can discriminate them.

\paragraph{Two admissible histories with identical boundary state.}
Initialize at $t_0$ in the product state
\[
\ket{\psi_0} \;=\; \ket{+}_S \otimes \ket{+}_E,
\qquad
\ket{+} := \tfrac{1}{\sqrt{2}}(\ket{0}+\ket{1}).
\]
Define a first interaction segment of duration $t_\theta$ with $\theta := g t_\theta$.

\begin{enumerate}
\item \textbf{History $H_A$ (no intermediate environment relabeling).}
Evolve under \eqref{eq:U_ZZ} for time $t_\theta$:
\[
\ket{\psi_A(B)} \;=\; U(t_\theta)\,\ket{\psi_0}.
\]

\item \textbf{History $H_B$ (environment relabeling within the inaccessible sector).}
Evolve identically for time $t_\theta$, then apply a local unitary on $E$,
\[
\ket{\psi_B(B)} \;=\; (\mathbb{I}_S \otimes \sigma_x^{(E)})\,U(t_\theta)\,\ket{\psi_0}.
\]
\end{enumerate}

Because $\sigma_x^{(E)}$ acts only on $E$, it does not affect the reduced state of
$S$ at the boundary:
\begin{equation}
\rho_S^{(A)}(B) \;=\; \mathrm{Tr}_E\!\big[\ket{\psi_A(B)}\!\bra{\psi_A(B)}\big]
\;=\;
\mathrm{Tr}_E\!\big[\ket{\psi_B(B)}\!\bra{\psi_B(B)}\big]
\;=\; \rho_S^{(B)}(B).
\label{eq:boundary_equal}
\end{equation}
A direct calculation gives
\begin{equation}
\rho_S(B)
\;=\;
\tfrac{1}{2}\Big(\mathbb{I} + \cos(2\theta)\,\sigma_x^{(S)}\Big),
\label{eq:rhoS_boundary}
\end{equation}
so the boundary state is mixed for generic $\theta$ and is \emph{identical} for
$H_A$ and $H_B$ within frame $F_S$.

\paragraph{Identical post-boundary evolution, distinct diagnostics.}
Now apply the \emph{same} post-boundary evolution block $C$ to both arms: evolve
under the same Hamiltonian \eqref{eq:H_ZZ} for time $t_\phi$ (with $\phi := g t_\phi$),
\[
\ket{\psi_{A,2}} \;=\; U(t_\phi)\,\ket{\psi_A(B)}, \qquad
\ket{\psi_{B,2}} \;=\; U(t_\phi)\,\ket{\psi_B(B)}.
\]
Let the diagnostic observable be the $X$-visibility on $S$,
\[
D(\phi) := \langle \sigma_x^{(S)} \rangle .
\]
Then one finds
\begin{equation}
D_A(\phi) \;=\; \cos\!\big(2(\theta+\phi)\big),
\qquad
D_B(\phi) \;=\; \cos\!\big(2(\phi-\theta)\big).
\label{eq:diagnostic_divergence}
\end{equation}
In particular, for generic $(\theta,\phi)$,
\[
D_A(\phi) \neq D_B(\phi)
\quad\text{even though}\quad
\rho_S^{(A)}(B)=\rho_S^{(B)}(B).
\]
Thus, within the operational frame $F_S$ (where only $\rho_S(B)$ is verified at
the boundary), \emph{operationally indistinguishable boundary states admit
distinct future extensions} under identical post-boundary evolution.

\paragraph{Symmetry group and the meaning of ``symmetry breaking.''}
In frame $F_S$, local unitaries on $E$ at the boundary are operationally
undetectable from $S$ alone and therefore generate an admissible-history
indistinguishability:
\[
G := \{ \mathbb{I}_S \otimes V_E \;|\; V_E \in \mathrm{U}(2) \}.
\]
By construction, $G$ acts trivially on the reduced boundary state, but can act
nontrivially on admissible continuations because $E$ participates in the fixed
post-boundary dynamics. Any state-augmentation of $S$ that distinguishes $H_A$
from $H_B$ at $B$ must encode which $V_E$ occurred, but that information is
\emph{not verifiable within $F_S$}. In the sense of
Definition~\ref{def:faithful-state-representation}, such an augmentation is
unfaithful relative to $F_S$: it distinguishes preparations that are
operationally indistinguishable under all measurements available at the boundary.

\paragraph{Interpretation.}
This example is fully compatible with global unitary quantum mechanics: the two
histories correspond to different global states on $SE$, hence different reduced
future behavior when $S$ remains coupled to $E$. The point of $\Sigma_2$ is
classificatory: within a fixed reduced operational frame, admissibility does not
factorize through the reduced state, and any attempt to restore sufficiency by
state enlargement requires representational enlargement beyond the frame (i.e.,
introducing variables that cannot be independently verified without accessing $E$).

The complete density-matrix derivations, explicit verification of $G$-equivalence, and the
demonstration that any state augmentation distinguishing $H_A$ from $H_B$ necessarily violates
operationally verified indistinguishability are given in Appendix~\ref{app:hamiltonian-derivation},
Sec.~\ref{subsec:symmetry-group}.

\subsection{Implications}

Proposition~\ref{prop:first-order-insufficiency} does not assert that first-order
descriptions are incorrect, nor that they fail empirically in all domains. It
establishes a boundary: there exist classes of constraints for which state-based
descriptions are formally incapable of expressing admissibility without structural
distortion.

This boundary motivates the introduction of a higher-order descriptive layer in
which constraints act directly on admissible histories rather than on instantaneous
states. The next section formalizes such a layer.

\section{\texorpdfstring{Second-Order Constraint Geometry ($\Sigma_2$)}{Second-Order Constraint Geometry (Sigma2)}}
\label{sec:definition-sigma2}

The limitations identified in
Sec.~\ref{sec:first-order-limits} point to a missing representational layer. When
admissibility depends on ordered history rather than instantaneous configuration,
constraints cannot be expressed faithfully within a state-based geometry. What is
required is not new dynamics, but a language in which such constraints can be stated
directly.

Second-order constraint geometry provides this language. 

Here “geometry” is used in the structural sense of the Erlangen program: $\Sigma_2$
classifies admissible-history structure by the invariants and equivalence relations
induced under symmetry actions and operational projections, without presupposing a
metric or manifold structure.

\subsection{Admissible-history space}

Let $\mathcal{H}$ denote the space of admissible histories of a system. Elements
$h \in \mathcal{H}$ are ordered sequences (or continuous curves) of states consistent
with the system’s underlying laws or constraints. No privileged parametrization,
orientation, or temporal direction is assumed at this level. Admissibility is
defined prior to any projection onto a reduced description.

\subsection{Second-order constraints}

A \emph{second-order constraint} is a restriction on $\mathcal{H}$ that cannot be
factorized into constraints on instantaneous states alone.

\begin{definition}[Second-order constraint]
A constraint $C$ is second-order if it is a predicate
\[
C : \mathcal{H} \to \{0,1\}
\]
such that there exists no function $c : \mathcal{S} \to \{0,1\}$ satisfying
\[
C(h) = \prod_{t} c(s_t)
\quad \text{(or any equivalent pointwise factorization)}
\]
for all histories $h = \{s_t\}$ in $\mathcal{H}$.
\end{definition}

Equivalently, a second-order constraint depends on relational properties between
distinct segments of a history rather than on instantaneous values.

\subsection{Second-order constraint geometry}

We define $\Sigma_2$ as the geometric structure induced by second-order constraints
on $\mathcal{H}$.

\begin{definition}[Second-order constraint geometry ($\Sigma_2$)]
The second-order constraint geometry $\Sigma_2$ is the space of admissible histories
$\mathcal{H}$ equipped with equivalence relations, exclusions, or admissibility
conditions defined directly on histories rather than on states.
\end{definition}

Geometrically, $\Sigma_2$ organizes $\mathcal{H}$ into regions, classes, or
topological sectors determined by admissibility relations that are invisible at the
level of instantaneous state descriptions.

\subsection{Relation to first-order dynamics}

$\Sigma_2$ does not replace first-order dynamics. Instead, it acts as a constraint
layer that restricts which histories generated by first-order laws are admissible.
A system may therefore satisfy first-order dynamical laws on $\mathcal{S}$ together
with second-order admissibility constraints on $\mathcal{H}$ without contradiction.

The distinction is not between dynamics and non-dynamics, but between constraints
acting on states and constraints acting on histories.

\subsection{Non-representability as state variables}

Second-order constraints generally cannot be represented as additional state
variables without introducing asymmetry or violating invariances of the admissible
history space. Any attempt to encode such a constraint as an instantaneous variable
requires selecting a representative from a class of histories, thereby importing
ordering or boundary information not intrinsic to $\mathcal{H}$. This is precisely
the failure mode identified in
Proposition~\ref{prop:first-order-insufficiency}.

\subsection{Scope and intent}

At this stage, $\Sigma_2$ is introduced as a formal extension rather than as a
physical hypothesis. No claim is made that a given physical system requires
second-order constraints, nor that such constraints generate novel dynamics. The claim is instead that when state sufficiency fails under symmetry-preserving representations, second-order constraints provide the minimal faithful description.

The purpose of $\Sigma_2$ is to provide a mathematically explicit language for
expressing admissibility structures that are otherwise forced into state-based
descriptions via implicit asymmetries, boundary selections, or hidden bookkeeping.
Subsequent sections will examine when and how such constraints become
operationally detectable.

