\section{Case Study: Measurement and Post-Selection}
\label{sec:case-measurement-postselection}

This section provides a concrete case study illustrating the necessity of
second-order constraint geometry $\Sigma_2$ within standard quantum-mechanical
practice.
The example concerns measurement and post-selection, chosen because it is
uncontroversial, operationally precise, and ubiquitous across quantum
experiments.

The analysis does not depend on interpretational commitments.
It relies only on standard measurement theory and experimentally realizable
protocols.

\subsection{Setup}

Consider a quantum system prepared in an initial state $\rho_0$ at time $t_0$.
At a later time $t_1$, a measurement is performed with outcomes labeled by
$i$, associated with measurement operators $\{M_i\}$ satisfying
$\sum_i M_i^\dagger M_i = \mathbb{I}$.

Upon obtaining outcome $i$, the system is assigned the post-measurement state
\[
\rho_i = \frac{M_i \rho_0 M_i^\dagger}{\mathrm{Tr}(M_i^\dagger M_i \rho_0)}.
\]

This update rule is standard and uncontroversial.

\subsection{State equivalence versus preparation equivalence}

Now consider an alternative preparation procedure in which the system is
directly prepared at time $t_1$ in the state $\rho_i$, without any prior
measurement or conditioning.

From the standpoint of quantum formalism, the two preparations are equivalent:
the system is described by the same density operator $\rho_i$, and all future
measurement statistics conditioned only on $\rho_i$ are identical.

However, the two preparations are not operationally equivalent in all contexts.

\subsection{Failure of admissibility equivalence}

Although the instantaneous states coincide, the admissibility of subsequent
operations may differ.

Examples include:
\begin{itemize}
\item extension of the system to a larger Hilbert space including an apparatus
      or environment,
\item reversal or partial reversal of the measurement interaction,
\item consistency of retrodictive inference about prior correlations.
\end{itemize}

In particular, a post-selected state $\rho_i$ obtained via measurement generally
does not admit a unitary extension that erases the measurement record without
access to additional degrees of freedom.
By contrast, a directly prepared state $\rho_i$ may admit such an extension.

This distinction cannot be recovered from the density operator alone.

\subsection{Second-order constraint interpretation}

From the standpoint of $\Sigma_2$, the difference between the two cases lies not
in the state but in the admissible histories leading to that state.

The post-selected ensemble is constrained by a conditioning boundary at the
measurement event.
Only trajectories consistent with outcome $i$ are admissible beyond that
boundary.

The directly prepared ensemble lacks this constraint.
Its admissible histories are not restricted by a prior measurement outcome.

Thus, the admissibility of future operations depends on ordered history, not
solely on instantaneous state.

The structural role of these measurement and post-selection examples can be made
explicit by relating them directly to the symmetry analysis in
Sec.~\ref{subsec:spin-half-symmetry-example}. In each case, admissible histories
that are operationally indistinguishable at the level of the reduced state
$\rho_S$ are related by transformations acting on degrees of freedom that are
inaccessible or uncontrolled from the perspective of the reduced description.
These transformations define an admissible-history symmetry: they act trivially
on instantaneous state but nontrivially on the space of histories by altering
which future operations (extensions, reversals, or conditional procedures)
remain physically realizable.

From this perspective, the distinction between directly prepared and
post-selected states is not merely a matter of system boundary specification.
Rather, it instantiates the same pattern identified in Sec.~3.5: a symmetry of
the admissible-history space arising from operational indistinguishability, whose
preservation prevents admissibility from factorizing through instantaneous state.
The failure of faithful state sufficiency here is therefore structural and
diagnostic, not an artifact of incomplete modeling or misuse of reduced states.

\subsection{Relation to irreversibility}

Having identified the symmetry-based origin of the admissibility distinction, this case study mirrors the structure identified in projection-induced arrow
analyses.
Measurement introduces a boundary-conditioned restriction on admissible
histories, even when subsequent dynamics are unitary.

Irreversibility arises not from intrinsic asymmetry in the dynamics, but from the
constraint imposed by conditioning.
This parallels the role of boundary selection in thermodynamic and decoherence
contexts.

\subsection{Why this is not an interpretational claim}

Nothing in this analysis asserts that quantum states are incomplete, hidden, or
epistemic.
The quantum formalism remains fully valid for predicting measurement statistics.

The point is classificatory:
state descriptions are sufficient for prediction, but not for admissibility.

$\Sigma_2$ provides a formal language for this distinction without modifying the
dynamics or adding variables.

\subsection{Summary}

Measurement and post-selection provide a minimal example in which:
\begin{enumerate}
\item identical quantum states arise from distinct preparation histories,
\item those histories impose different admissibility constraints,
\item the difference is not encoded in the instantaneous state.
\end{enumerate}

This establishes the necessity of second-order constraint structure in at least
one central domain of quantum practice.