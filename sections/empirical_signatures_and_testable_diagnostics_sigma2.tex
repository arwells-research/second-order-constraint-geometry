\section{\texorpdfstring{Empirical Signatures and Testable Diagnostics of $\Sigma_2$}
{Empirical Signatures and Testable Diagnostics of Sigma2}}
\label{sec:empirical-signatures-sigma2}

The purpose of introducing $\Sigma_2$ is not to add interpretive structure, but to
identify a class of constraints that are empirically distinguishable from
state-based descriptions. The framework has empirical content only insofar as it
yields observable signatures that separate first-order, state-sufficient models
from second-order, trajectory-constrained ones.

This section identifies such signatures and clarifies the conditions under which
$\Sigma_2$ structure would be supported or ruled out.

\subsection{Failure of state sufficiency as an empirical criterion}

The defining empirical signature of $\Sigma_2$ is the failure of state
sufficiency. Operationally, this failure manifests when two preparations that are
indistinguishable at the level of instantaneous state yield systematically
different statistics solely due to differences in admissible history.

Formally, a system exhibits second-order constraint structure if there exist
histories $h_1, h_2$ such that
\[
\Pi(h_1) = \Pi(h_2)
\quad\text{but}\quad
\mathbb{E}[Q \mid h_1] \neq \mathbb{E}[Q \mid h_2],
\]
for some operational projection $\Pi$ and diagnostic quantity $Q$.

This condition is both necessary and sufficient for detecting $\Sigma_2$:
state-based descriptions predict identity of outcomes under identical
instantaneous descriptions, whereas second-order constraints permit divergence
driven by admissible history alone.

\subsection{Postselection asymmetry}

Postselection provides a direct probe of admissible-history dependence. In a
state-sufficient framework, conditioning on future outcomes is operationally
equivalent, up to normalization, to conditioning on past preparations. Any
asymmetry must therefore be attributable to explicit boundary conditions or
measurement disturbance.

By contrast, $\Sigma_2$ predicts a stronger effect. Distinct postselection
criteria applied to ensembles that share identical reduced states at an
intermediate time may yield diagnostic statistics that cannot be reconstructed
from any mixture of forward-prepared states with the same instantaneous
description. When such divergence persists after controlling for disturbance and
selection bias, it indicates that admissibility depends on trajectory-level
structure rather than on state alone.

\subsection{History-conditioned decoherence}

Standard decoherence theory predicts rates determined by instantaneous
system–environment coupling and state. Once these are fixed, no further dependence
on prior history is expected.

Second-order constraints alter this expectation. If admissibility depends on the
ordering or structure of prior interactions, then decoherence behavior may vary
even when instantaneous couplings and reduced states are identical. Observable
differences in decoherence envelopes, revival structure, or coherence decay under
matched state and Hamiltonian conditions therefore constitute a diagnostic
signature of $\Sigma_2$ structure.

Such effects, if observed under controlled conditions, cannot be captured by
purely state-based decoherence models without introducing asymmetry or hidden
bookkeeping.

\subsection{History dependence without stored state}

A common response to observed history dependence is to posit hidden memory
variables or to enlarge the state space so as to restore Markovianity. $\Sigma_2$
predicts a qualitatively different form of dependence: effects that track
admissible history without corresponding to any localized or dynamically evolving
memory register.

Empirically, this distinction can be tested by excluding internal memory
mechanisms—through architectural constraints, isolation, or reset procedures—
while retaining observable history dependence. Persistence of such effects under
these conditions indicates irreducible trajectory-level constraint structure
rather than unmodeled state.

\subsection{Order-sensitive protocol dependence}

Second-order constraints imply that the ordering of operations matters even when
their net action on instantaneous state does not. Two protocols may implement
identical overall transformations and yield identical final-state statistics,
yet differ in diagnostic outcomes solely due to the ordering of intermediate
admissible steps.

Such order sensitivity is forbidden in genuinely state-sufficient frameworks, in
which admissibility factorizes through state. Its observation therefore provides a
direct and operationally accessible signature of $\Sigma_2$ constraints.

\subsection{Relation to existing foundational tests}

Many existing experiments already exhibit behavior consistent with
second-order constraints, though they are typically interpreted in different
terms. Bell inequality violations, contextuality tests, and delayed-choice
experiments can be re-expressed as probes of admissible-history structure rather
than as statements about nonlocal or contextual state properties.

Importantly, $\Sigma_2$ does not predict stronger violations than quantum theory.
Its claim is structural: such violations persist even when attempts are made to
restore state sufficiency through auxiliary variables, retrocausal models, or
state-based reinterpretations. The persistence itself is the diagnostic signal.

\subsection{Falsifiability}

The $\Sigma_2$ framework is falsified if all observed history-dependent effects can
be eliminated by a symmetry-preserving, state-local reformulation. If expanding
the instantaneous state space, introducing explicit memory variables, or redefining
operational projections suffices to restore full state sufficiency without
residual ordering or boundary dependence, then second-order constraints are
unnecessary.

The claim is therefore sharp: some physical systems exhibit irreducible history
dependence that cannot be captured by state alone. If no such systems are found,
$\Sigma_2$ should be rejected.