\section{Limits of State Sufficiency}
\label{sec:limits-state-sufficiency}

A defining assumption of standard physical formalisms is that the instantaneous
state of a system provides a sufficient description for all future admissible
evolution. In classical mechanics this role is played by phase-space points; in
quantum mechanics by state vectors or density operators. Under this assumption, fixing the state fixes the space of physically realizable futures.

This section identifies the precise sense in which this assumption holds, and the
structural conditions under which it fails. The failure is not epistemic or
approximate, but geometric: it arises when admissibility depends on ordered
history rather than instantaneous configuration.

\subsection{State sufficiency as a conditional principle}

We say that a system admits a \emph{state-sufficient description} if there exists
a state space $\mathcal{S}$ such that, for any two admissible histories
$h_1,h_2$,
\[
\Pi(h_1,t)=\Pi(h_2,t) \;\Rightarrow\;
\mathcal{F}(h_1,t)=\mathcal{F}(h_2,t),
\]
where $\Pi(h,t)$ denotes the instantaneous state at time $t$ and
$\mathcal{F}(h,t)$ denotes the set of admissible future continuations.

Under this condition, the instantaneous state determines not only future
measurement statistics, but the full space of physically realizable operations.
This requirement is stronger than predictive completeness: it asserts that
admissibility itself is state-local.

Standard formulations of mechanics implicitly assume this form of sufficiency.
The assumption is not trivial; it is a substantive constraint on how physical
structure may be represented.

\subsection{Prediction versus admissibility}

It is useful to distinguish two roles played by physical states. First, a state
may be \emph{predictively sufficient}, in the sense that it determines the
probabilities of future measurement outcomes. Second, it may be
\emph{operationally sufficient}, in the sense that it determines which future
transformations, reversals, or control operations are physically admissible.

Quantum mechanics guarantees predictive sufficiency for isolated systems.
However, predictive sufficiency does not entail operational sufficiency. A
reduced state may correctly predict all future statistics while failing to encode
which operations remain realizable. The distinction becomes visible precisely
when control, reversibility, or extension of the system is considered.

\subsection{When state sufficiency fails}

State sufficiency fails whenever admissibility depends on relations between
distinct segments of a history rather than on instantaneous configuration.
In such regimes, two systems sharing the same state may nevertheless differ in
the set of physically realizable futures.

This situation arises in familiar contexts: when correlations are dispersed into
inaccessible degrees of freedom, when conditioning or post-selection restricts
which trajectories remain admissible, or when coarse-graining discards
information relevant to control or reversal. In each case, the common feature is
that admissibility is determined by ordered history rather than by state.

\subsection{Formal characterization}

Let $\mathcal{H}$ denote the space of admissible histories and
$\Pi:\mathcal{H}\to\mathcal{S}$ a state projection. State sufficiency requires
that $\Pi$ induce equivalence classes that are closed under future extension:
\[
\Pi(h_1)=\Pi(h_2)
\;\Rightarrow\;
\mathcal{C}^+(h_1)=\mathcal{C}^+(h_2),
\]
where $\mathcal{C}^+(h)$ denotes the admissible future continuations of $h$.

When this condition fails, the state space $\mathcal{S}$ is insufficient to
represent the system's full admissibility structure. The failure is structural: it
reflects the fact that admissibility is not factorizable into instantaneous
properties.

\subsection{Second-order constraints}

In regimes where state sufficiency fails, admissibility is governed by constraints
on \emph{histories} rather than on states. These constraints act on ordering,
accessibility, and composability of processes. They cannot be represented as
predicates on instantaneous configurations without loss or distortion.

The resulting structure is naturally described by a second-order constraint
geometry $\Sigma_2$, whose elements are subsets of $\mathcal{H}$ closed under
allowed transformations but not reducible to state equivalence classes. $\Sigma_2$
does not modify dynamics or introduce new variables; it provides a language for
constraints that are already operative but not state-local.

\subsection{Relation to irreversibility and arrows}

Many arrow-like phenomena arise precisely in regimes where state sufficiency
fails. The apparent loss of reversibility reflects not a change in microscopic
laws, but a restriction on admissible histories imposed by environmental
coupling, conditioning boundaries, or control limitations.

This clarifies why arrow-like behavior can coexist with time-reversal symmetric
dynamics: the asymmetry lies in admissibility rather than in the equations of
motion. The connection will be developed further in relation to
projection-induced arrow constraints.